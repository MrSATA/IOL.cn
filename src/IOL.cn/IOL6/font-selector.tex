\usepackage{fontspec}
\usepackage{xeCJK}
\usepackage{ifplatform}

% Load Computer Modern Unicode.
% As this file will be copied to work path, we use ../../../ here.
\setmainfont[
  Path = ../../../fonts/CMUnicode/,
  Extension = .ttf,
  BoldFont = *-Bold,
  ItalicFont = *-Italic,
  BoldItalicFont = *-BoldItalic,
  SlantedFont = *-RomanSlanted,
  UprightFont = *-Roman
]{CMUSerif}
\setsansfont[
  Path = ../../../fonts/CMUnicode/,
  Extension = .ttf,
  BoldFont = *-Bold,
  ItalicFont = *-Oblique,
  BoldItalicFont = *-BoldOblique,
  UprightFont = *
]{CMUSansSerif}

% Load Chinese Fonts.
% Due to the license beetween SinoType and Apple, we couldn't use SinoType's fonts under Windows or Linux.
% If non-Mac users contribute to our project, they should add proper fonts by themselves.
% Note that "-shell-escape" should be used for XeLaTeX. For details, check out ifplatform's documentation.

\ifmacosx
\setCJKmainfont[
  BoldFont = FZZCYSK--GBK1-0,
  ItalicFont = Kaiti SC Regular,
  BoldItalicFont = Kaiti SC Bold,
  SlantedFont = STFangsong
%]{Songti SC Regular}
]{FZXSSK--GBK1-0}
\setCJKsansfont[
  BoldFont = Heiti SC Medium,
  ItalicFont = Kaiti SC Regular,
  BoldItalicFont = Kaiti SC Bold
]{Heiti SC Light}
\else
\fi

% Use Computer Modern Serif for those characters.
\normalspacedchars{•‘’“”-—–̄́`ˇ}
\xeCJKsetup{CJKecglue={\hskip 0.1em}}
