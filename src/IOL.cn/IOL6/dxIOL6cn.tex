\def \thisling{中文}
\def \thislang{中文}
\def \thistext{中文文本}
\def \olympiad{第六届国际理论、数理及应用语言学奥林匹克竞赛}
\def \pgheader{第六届国际语言学奥林匹克竞赛}
\def \bulgaria{保加利亚}
\def \sunbeach{阳光海滩}
\def \olydates{2008年8月4 — 9日}
\def \probindl{个人赛题目}
\def \solsindl{个人赛解答}
\def \probteam{团体赛题目}
\def \soluteam{团体赛解答}
\def \probword #1{题 \##1}
\def \pontword{分}
\def \rulesmot{规则}
\def \regulats{解答规则}
\def \regulatx{毋需抄题. 将不同问题的解答分述于不同的答题纸上. 每张纸上注明题号、座位号和姓名}
\def \towarrant{否则答题纸可能被误放或遗失.}
\def \regulaty{解答需详细论证. 无解释之答案, 即便完全正确, 也会被处以低分.}
\def \editorsz{编者}
\def \edinchef{主编}
\def \goodluck{祝你好运}
\def \answersp{答案}
\def \fordinto #1{翻译成#1}
\def \outdrehu #1{翻译#1}
\def \corrcorr{找出正确的对应关系}
\def \transcri{转写这些词}
\def \writorth{用 Listuguj 正字法拼写}
\def \macschwa #1{在本{\tscr}中, #1 = 普通话 \word{横} (\word{héng})中的\word{e}}
\def \zoqwedge{$\approx$ 英语 \word{but} 中的 \word{u}}
\def \oshiroko{是开元音 \word{o}}
\def \eshiroko{$\approx$ 英语 \word{aspen} 中的 \word{a}}
\def \onororth #1#2#3{#1 $\approx$ 英语 \word{cat} 中的 \word{a}, #2 = 汉语拼音 \word{üe}. #3 是 #2 的短音}
\def \owumlaut{= 汉语拼音 \word{üe}}
\def \chumlaut{= French \word{eu} and \word{u} (German \word{ö} and~\word{ü})}
\def \uyumlaut{= 汉语拼音 \word{ü}}
\def \nbsuperw{表示前一个辅音圆唇}
\def \ruleforw #1#2{The letter #1 stands for a rounding of the lips after a consonant and for the sound #2 otherwise}
\def \rquvular #1#2{字母 #1 约等于汉语拼音 \word{h} 的浊音, 字母 #2 代表一个类似 \word{k} 且在同一部位发声的音}
\def \mandcons #1#2#3#4{#1 and~#2 are hard consonants similar to English \word{sh} in \word{shut} and \word{ch} in \word{chuck}, #3 and~#4 are soft consonants similar to \word{sh} in \word {sheet} and \word{ch} in \word{cheat}}
\def \zoqconsz #1#2{#1 = 汉语拼音 \word{z} 的浊音, #2 = \word{sh}}
\def \cjmicmac #1#2{#1 = 英语 \word{church} 中的 \word{ch}, #2 = 英语 \word{judge} 中的 \word{j}}
\def \spunvoid{是特殊的非浊辅音}
\def \velarnas{= 普通话 \word{杭} (\word{háng})中的\word{ng}}
\def \palatnas{$\approx$ 普通话 \word{娘} (\word{niáng})中的\word{ni}}
\def \jotisjot{nay}
\def \jotsound{= 汉语拼音 \word{y}}
\def \tsaffric{= 汉语拼音 \word{z}}
\def \frivelar{= 汉语拼音 \word{h}}
\def \dittovoi{是其浊音}
\def \glotstop{是一个特殊的辅音 (声门{\stp})}
\def \retrflex{$\approx$ 英语 \word{word} 和 \word{art} 中的 \word{d} 和 \word{t}, 发音时卷舌}
\def \thorneth #1#2#3{分别为英语 #2 和 #3 中的 #1}
\def \singsyll{的发音为单音节}
\def \hetteken{标记}
\def \aspirath{indicates that the preceding {\stp} consonant is aspirated (pronounced with a puff of air)}
\def \longmark{表示长元音}
\def \apostrof #1{The apostrophe indicates length if it follows a vowel, and is read as #1 if it follows a consonant}
\def \schwason #1{is pronounced, though not written, between any consonant and a following sonorant consonant #1}
\def \schwabis{is also pronounced before a consonant cluster at the beginning of a word}
\def \rulestop #1#2#3{#1 are pronounced as \voicon{}s (#2) at the beginning of a word or between vowels and as voiceless consonants (#3) at the end of a word or next to another consonant}
\def \incanada{在加拿大}
\def \northcan{在加拿大北部}
\def \inchiapa{在墨西哥南部的恰帕斯州}
\def \spokenca #1#2{#2, 约有#1人使用该语言}
\def \drehname{利富语}
\def \drehspok #1{在新喀里多尼亚东部的利富岛上, 超过#1人在使用利富语}
\def \indrehus{在利富语中}
\def \cemuname{卡穆希语}
\def \cemuspok #1{在新喀里多尼亚东海岸, 约有 #1 人使用卡穆希语}
\def \inukname{因纽特语}
\def \geninukt{因纽特语 (加拿大因纽特语) 属于爱斯基摩-阿留申语系}
\def \zoquenom{索克语 (Copainalá 变体) }
\def \genzoque #1{#1属于米塞-索克语系}
\def \bothaune{两种语言均属于南岛语系}
\def \gemicmac{米克马克语是一种阿尔冈昆语}
\def \inmicmac{下面是以被称为 Listuguj 的正字法拼写地米克马克语的单词, 它们的语音{\tscr}及汉语翻译}
\def \indrecem{下面是新喀里多尼亚两种语言的单词和词组}
\def \giveword #1{下面是#1的单词}
\def \inscalds #1{下面是古诺尔斯语诗歌的四个片段, 编写于公元九百年前后. 它们均遵循韵律 #1 (字面意义 ‘宫廷韵律’)}
\def \insentin #1{下面是#1的句子}
\def \redrecem #1#2{这里还给出了几个#1单词的#2翻译}
\def \dqstanza{下面给出一节诗, 其中十三个词被删去}
\def \theirtra{其汉语翻译}
\def \chaotict{ (乱序排列)}
\def \cdaskone #1#2#3#4{你觉得#2单词 #1 以及 #4单词 #3 是什么意思}
\def \cdasktwo #1#2#3#4#5{在#1中, #2 是指 ‘#3’, #4 是指 ‘#5’}
\def \allitera{头韵}
\def \videstat{See the statement of the problem}
\def \allitert #1{#1 的一个主要规则是头韵}
\def \alliterh{每个押韵对句 (行对) 的第一行包含两个单词, 首音相同, 且第二行的第一个单词亦是由此音开头}
\def \vocallit #1{元音被认为与另一个元音或 #1 合头韵}
\def \nunirule{但这不是唯一的规则}
\def \mdiffers{不止一份上述文本的手抄本传承下来}
\def \mvariats{有时候在文本相同的地方出现了不同的单词, 学者需判断哪种变体为原文}
\def \huchoose{不同的考虑可能影响结论. 有时候诗律规则有助于确定某些错误变体}
\def \nonlybut #1#2#3{比如, 在行 #1 中我们不仅仅发现 #2, 也发现了 #3}
\def \outother #1#2{可以根据诗的结构排除, 但 #1 和 #2 都可接受, 因此人们需要其他理由来选择单词.}
\def \drotqtri #1#2#3{在行 #1 中, #2 和 #3 均出现在手抄本中, 但 #3 并不满足诗的要求.}
\def \vadrules #1{描述在一对 #1 的押韵对句中观察到的规则}
\def \dqfiller{下面的列表包含了 (按字母序排列) 全部 13 个删去的词以及不属于诗节 V 的两个单词}
\def \fillgaps{填补诗节 V 中的空}
\def \wwonorse{古诺尔斯语是一种北日耳曼语,在公元七百年至一千一百年左右使用}
\def \onornorm{本题的全部诗篇均以规范化的正字法给出并符合这一体裁的规则}
\def \qtysylla{Number of syllables. Each line contains 6 syllables}
\def \inrhymes{Internal rhyme}
\def \inrhymep{Let us denote the vowels (and diphthongs) in each line by}
\def \inrhymeq #1#2{At least one consonant immediately following V$_5$ must immediately follow #1 #2}
\def \inrhymer{Also, in even lines}
\def \egdrotts #1{For instance, cf.\ lines #1 (alliteration is marked in boldface, internal rhyme by underlining)}
\def \excesswd{Leftover words}
\def \autelexp{圣地是教堂最为重要神圣的部分}
\def \autelmot{圣地}
\def \abeilles{一群蜜蜂}
\def \abeillex{蜜蜂}
\def \abeimult{蜜蜂}
\def \animalwd{动物}
\def \aquagran{水}
\def \banamult{香蕉}
\def \bananasz{一串香蕉}
\def \bananawd{香蕉}
\def \biblemot{圣经}
\def \boiremis{饮用}
\def \bonemult{骨头}
\def \boneword{骨头}
\def \calendar{日历}
\def \carrelet{钻子}
\def \costemot{海岸}
\def \dimanche{星期天}
\def \dormimot{睡觉}
\def \ecclesia{教堂}
\def \esperonx{踢马刺}
\def \esperony{戳动物的工具}
\def \fourchet{叉子}
\def \frontier #1{#1的边界}
\def \holyhome{神圣的}
\def \holyjour{神圣的}
\def \holypost{神圣的}
\def \holytome{神圣的}
\def \homegran{房子}
\def \homeword{房子}
\def \jourmult{日子}
\def \jourword{日子}
\def \kurkalem{铅笔}
\def \litparol{床}
\def \multitud #1{#1的复数}
\def \murparol{墙}
\def \nuitgran{夜晚}
\def \piquemis{用来戳的}
\def \piquemot{戳}
\def \tool #1{#1工具}
\def \postdorm{睡觉的地方}
\def \postword{地方}
\def \schrimis{用来写的}
\def \schrimot{写}
\def \skeleton{骨架}
\def \tomeword{书}
\def \twilight{黄昏}
\def \verremot{杯子}
\def \shamanex{萨满是某些文化中的祭祀, 男巫以及治疗师}
\def \thejager{这个猎人}
\def \deshaman{这个萨满}
\def \sinukoer{你的狗}
\def \inuktitq{你来过了}
\def \inuktitk{你摔倒了}
\def \inuktitl{你伤了你自己}
\def \shottest #1{你射了#1}
\def \inuktitr{你治好了一个猎人}
\def \inuktitc{你刺了那条狗}
\def \adogword{一条狗}
\def \tteacher{这个老师}
\def \inuktitj{治好了一个医生}
\def \inuktitw{伤了一个老师}
\def \inuktits{伤了你}
\def \inuktitb{看见过你}
\def \inuktitg{刺了那匹狼}
\def \inuktiti{射了一只北极熊}
\def \inuktitf{这个男孩射了那个医生}
\def \inuktite{这条狗伤了你的老师}
\def \inuktitm{这匹狼看见过你的萨满}
\def \inuktitu{你的狼摔倒了}
\def \inuktith{这只北极熊来过了}
\def \inuktitn{你的北极熊伤害了一个男孩}
\def \inuktita{这个医生治好了你}
\def \inuktitd{你的医生治好了你的男孩}
\def \inuktito{你的猎人治好了他自己}
\def \inuktitt{这位老师看到了那个男孩}
\def \withword{`with'}
\def \forparol{for}
\def \abovemot{上方}
\def \renderpl{plural}
\def \zadiente{为了这颗牙齿}
\def \zacourge{为了这颗小果南瓜}
\def \courges{一些小果南瓜}
\def \cucourge{带着南瓜小果}
\def \nakpat{一株仙人掌}
\def \kittel{一个水壶}
\def \zakittels{为了这些水壶}
\def \sukumguy{在这座小镇上方}
\def \zakumguys{为了这些小镇}
\def \balkan{一座山}
\def \balkanes{群山}
\def \cumokmok{带着小麦}
\def \cumikittel{带着我的水壶}
\def \mimokmok{我的小麦}
\def \mipillar{我的柱子}
\def \sumihimmel{在我的这片天空上方}
\def \supillar{在这些柱子上方}
\def \pillars{这些柱子}
\def \tudientes{你的牙齿 (复数)}
\def \cutunakpat{带着你的仙人掌}
\def \tukumguy{你的小镇}
\def \tucaycay{你的葡萄藤}
\def \himmel{一片天空}
\def \poxkuy{一把椅子}
\def \zapoxkuy{为了这把椅子}
\def \supoxkuys{在这些椅子上方}
\def \zacaycay{为了这株葡萄藤}
\def \sucaycay{在这株葡萄藤上方}
\def \shadow{一个影子}
\def \shadows{许多影子}
\def \sushadows{在这些影子上方}
\def \cushadow{带着这个影子}
\def \axeparol{斧头}
\def \unsafewd{不安全}
\def \arxangel{天使长}
\def \shoeword{钉蹄铁 \emph{(给马)}}
\def \spoonmot{匙}
\def \agentmot{印第安代理人}
\def \lawparol{法律}
\def \lieontop{躺在顶部}
\def \indianfe{印第安女人}
\def \oheadmot{上方, 在头顶上}
\def \stovemot{炉}
\def \foolishn{愚蠢 (名词)}
\def \springwa{泉水}
\def \raspberr{树莓}
\def \worships{崇拜}
\def \goatword{山羊}
\def \southmot{南方}
\def \snakemot{蛇}
\def \lookroon{环顾四周}
\def \therefor{所以, 因此}
\def \defobj #1{definite object}
\def \cedreord{The modifier follows its head in both languages}
\def \instruct{The Inuktitut sentences have the following general structure}
\def \wherinuk #1#2{where #1 are nouns and #2 is the verb}
\def \declinuk #1#2#3{If a noun gets the ending #1 when it is either a \defobj{} or a subject of a sentence that doesn't have a \defobj{}, it also gets #2 before the ending #3 when it is an in\defobj{}}
\def \sayyours #1#2#3#4{To say `your', #1 is replaced by #2, #3 by #4}
\def \zoquesfx{The noun suffixes seen in this problem are}
\def \suffinuk{The verb receives the following suffixes}
\def \jortinuk #1#2{#1 following a vowel or #2 following a consonant}
\def \endingin{an ending for the persons of the subject and the \defobj{}, if there is one}
\def \schemata{in the first two schemata}
\def \schematr{in the third schema}
\def \transref{一个没有宾语的及物动词当作反身动词处理}
\def \simnasal #1#2#3{After a nasal consonant (#1) the {\stp}s #2 become voiced (#3 respectively)}
\def \metathyk #1#2{If #1 comes after #2, the two sounds exchange places}
\def \possezoq #1#2{The possessive pronouns are #1 `my' and #2 `your'}
\def \prenasal{if the noun begins with a \stp, this consonant becomes voiced and the corresponding nasal appears before it}
\def \hh{送气}
\def \stp{塞音}
\def \frc{擦音}
\def \tscr {转写}
\def \transcrn{\tscr}
\def \fqts #1{反切\tscr#1}
\def \risetone{升}
\def \flattone{平}
\def \falltone{降}
\def \altotone{高}
\def \bajotone{低}
\def \Guangyun{\textsl{广韵}}
\def \xarakter{汉字}
%\def \chinname{Chinese}
\def \mandname{普通话}
\def \cantname{广东话}
\def \putongww{普通话是中国的官方语言, 基于北京方言. 约八亿五千万人使用该语言}
\def \wuandyue{九千万人使用吴语 (上海话), 七千万人使用广东话 (粤语).}
\def \whencomp #1{在编译#1时}
\def \homochin #1#2{在编译#1字典时 (#2), 汉语相当同质化}
\def \nophonet{由于汉字不是表音文字}
\def \oneastwo{该字典使用了一套简单的系统, 通过两个汉字来表示一个汉字的发音, 而读者理应知道前者的发音 (它们是常用字)}
\def \calledfq{这套系统叫做\emph{反切}.}
\def \hetechin{后来, 虽然汉语方言分化, 但许多古代的\fqts{}仍可使用, 只不过在不同的方言中有不同 (且更复杂) 的使用方法.}
\def \herecant{下面是一些反切{\tscr}. 每个汉字的广东话读音亦给出.}
\def \heremand{下面是另外一些{\tscr}, 但只给出了其普通话读音}
\def \canmando{下面是些汉字及其广东话和普通话的读音}
\def \canmanre{下面是另外一些汉字及其广东话和普通话的读音}
\def \hucanton{解释古代\fqts{}是如何应用于现代广东话的}
\def \huwasuse #1{#1\fqts{}是如何工作的}
\def \whatonly #1{在广东话中, 上述{\tscr}只有一个可以应用这条简单的老规则来得到正确的结果. 哪一个呢?}
\def \nownovoi #1{在多数当代汉语方言中 (包括广东话和普通话) 不存在\voicon{}, 除了响音 (#1)}
\def \ccvoiced #1{在#1编译时汉语存在其他\voicon{}, 它们后来并成对应的清音}
\def \fromvoid{浊}
\def \voicon #1{浊辅音}
\def \tounvoid{清}
\def \devofric #1#2{#1{\frc}变成#2{\frc}}
\def \devostop #1#2{#1{\stp}变成{\hh}或不{\hh}#2{\stp}}
\def \wuvoiced{浊音在吴语中得到了保留}
\def \wuvoicex #1#2#3#4{比如, 汉字 #1 在吴语中发作 #2, 在广东话中发作 #3, 在普通话中发作 #4}
\def \wherevoi #1{上一节中的哪些汉字#1首辅音为浊辅音}
\def \whyvdasp{这些\voicon{}在广东话中变得{\hh}或者不{\hh}取决于什么条件?}
\def \tredecar{在古汉语中存在四种声调, 但是只有三种出现在此题中}
\def \hutresex{解释这三种音调是如何演变出广东话的六种音调.}
\def \formanda{暂时忽略音调, 给出在普通话中使用古代\fqts{}的规则.}
\def \tonemiss{一些音调被移除}
\def \devemand{描述音调和浊首辅音是如何在普通话中演变的}
\def \humandar{可以总结出哪些在普通话中读\fqts{}的音调的规则?}
\def \rarewhat{一些初始的辅音和音调组合在现代普通话中极为罕见. 哪些呢?}
\def \supptone{判断出遗失的音调是哪些}
\def \readfoll #1{给出下面的{\tscr}#1的读法}
\def \incanton{在广东话中}
\def \inmandar{在普通话中}
\def \infohave{一些{\tscr}本不可读, 但这道题包含了足以读出它们的信息}
\def \triparts{汉语音节由三部分组成}
\def \onrimton{声母 (首辅音, 可能不存在如 3B), 韵母 (后面的所有音) 和声调}
\def \cantones #1#2{广东话音调可以认为存在两种不同的性质: 音高 (#1) 和轮廓 (#2).}
\def \socanton{若要在广东话中使用\fqts{}, A 的声母和声调音高将于 B 的韵母及声调轮廓组合}
\def \decanton #1#2#3{But if A’s (and X’s) tone is #1, X’s onset, if a \stp, must always be {\hh} if B’s (and X’s) tone is #2, and un{\hh} if it is #3}
\def \onsarhyb{Certainly the onset was from the A character, and the rhyme from B}
\def \aspirodd{But the aspiration rule is strange}
\def \noterken{Probably it was not part of the original fanqie system}
\def \tonfromb{Maybe the tone came from only one of the two characters? That has to be B, because the old rule should give correct results in only one {\tscr}.}
\def \urfanqie{Thus the original simple rule for fanqie was: A’s onset is combined with B’s rhyme and tone}
\def \onlyelve{Only {\tscr} 11 can be read now using this rule}
\def \sonorlow #1{Looking at the syllables with a sonorant onset, we see that they are always in a low tone (#1)}
\def \voitolow{Assuming that all \voicon{}s evolved alike in Cantonese, we may conclude that what is in a low tone now, had a voiced onset earlier}
\def \volosupp{This is also true of the character of the example from Wu. What is said in \textbf{(d)} supports this idea.}
\def \sherevoi{Thus the characters whose onsets were voiced are}
\def \ifiatall{if it had an onset at all}
\def \voidevel #1#2{Voiced {\stp}s became {\hh} if the tone was #1, and un{\hh} if it was #2}
\def \contoura{The contours of the Cantonese tones correspond to the three tones of Classical Chinese}
\def \nuheight{tone height is an innovation brought about by the evolution of the \voicon{}s.}
\def \explaict #1{因此我们可以解释为什么\fqts{}#1这样读}
\def \samehigh{The X character has the same tone height as A because it got its onset from A, and height in Cantonese is determined by the voicing of the onset in Classical Chinese}
\def \otherasp{But if the onset was a voiced \stp, it could evolve in different ways in X and A, because its aspiration was determined by the tone contour, which X got from B, and it could differ from A’s contour.}
\def \mandcomp{In Mandarin onsets and rhymes are not combined in such a straightforward way as in Cantonese}
\def \conoccur #1#2#3{It can be noted that after #1 we always find #2, whereas #3 are never followed by these vowels}
\def \dejaknow{We already know that the onset came from A and the rhyme from B}
\def \whenrest{When the constraint above came into being}
\def \ruharden #1#2#3#4{#1 was lost and #2 became #3 after #4}
\def \rusoften #1#2#3{#1 became #2 before #3}
\def \usethese #1{#1使用\fqts{}同样需要应用如上规则}
\def \onsetcis #1#2{if #1's onset is #2 and}
\def \rhymebei #1#2{B's rhyme starts with neither #1 nor #2}
\def \onsetane{A's onset is none of these}
\def \indeterx #1{we can't determine what X's #1 is}
\def \acconset{onset}
\def \accrhyme{rhyme}
\def \vocanton{On the basis of the tone of the Cantonese syllable we can determine whether the onset was voiced or not in Classical Chinese}
\def \mandtone{In Mandarin the tones developed as follows}
\def \mandtons #1#2#3{#2 if the onset was #1, #3 otherwise}
\def \voinoson{voiced but not a sonorant}
\def \unvoiced{voiceless}
\def \decontur{We see that the contour is not preserved here}
\def \somanton #1{In \fqts{s} read #1 the tones work as follows}
\def \legecons #1#2#3#4{Here #1 stands for a sonorant, #2 for a \frc, #3 for an un{\hh} and #4 for an {\hh} \stp}
\def \dimanton{Thus most of the time X's tone in Mandarin can't be derived unambiguously from A's and B's tones, though in some cases it can.}
\def \rarethat{Syllables with a sonorant onset and tone 5 or with an un{\hh} onset and tone 35 should not exist in Mandarin (if they do, then the rules must have had exceptions).}
\def \chintons{每个汉语方言都有固定数目的音调 (每个音节发音的旋律)}
\def \chaotons{本题使用了语言学家赵元任提出的系统, 其用数字 1 (最低) 到 5 (最高) 来标记音高的五级, 并将音调转写成连续的音级}
\def \havetons{你所需的全部音调都在本题中出现了}
\def \zidenote{如果你不想写汉字, 你可以使用{\tscr}的序号并指明具体的汉字来指代它们: X (被转写的)}
\def \nrintscr #1{#1用于{\tscr}的汉字}
\def \premiere{第一个}
\def \deuxieme{第二个}
\def \nnovowel #1{注意普通话的 #1 汉字的读法不包含元音}
\def \enquetea{你做了哪些题目?}
\def \enqueteb{你最喜欢哪道题?}
\def \enquetec{你觉得哪道题最难?}
\def \enqueted{你觉得哪道题最简单?}
\def \keredomo{However,}
\def \alwaysso{always}
\def \eg{如}
\def \et{和}
\def \au{或}
\def \Vhimself{\textsf{V} (himself)}
\def \aYrender{a \textsf{Y}}
\def \theYrend{the \textsf{Y}}
\def \solo{仅仅}
\def \just{就}
\def \rite{恰好}
\def \como{好像}
\def \cosi{像是}
\def \ABname{Alexander Berdichevsky}
\def \BBname{Bozhidar Bozhanov}
\def \SBname{Svetlana Burlak}
\def \DGname{Dmitry Gerasimov}
\def \XGname{Ksenia Gilyarova}
\def \IGname{Ivaylo Grozdev}
\def \SGname{Stanislav Gurevich}
\def \IDname{Ivan Derzhanski}
\def \AHname{Adam Hesterberg}
\def \BIname{Boris Iomdin}
\def \IIname{Ilya Itkin}
\def \RPname{Renate Pajusalu}
\def \APname{Alexander Piperski}
\def \MRname{Maria Rubinstein}
\def \LFname{Ludmilla Fedorova}
\def \TTname{Todor Tchervenkov}
\def \LMSname{刘闽晟}
\def \edinames{\ABname, \BBname, \SBname, \IDname\ (\edinchef), \LFname, \DGname, \XGname, \IGname, \SGname, \AHname, \BIname, \IIname, \RPname, \APname, \MRname, \TTname}
\def \whowroti{\LMSname}
\def \whowrotj{\LMSname}
