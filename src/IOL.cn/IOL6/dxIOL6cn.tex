\def \thisling{中文}
\def \thislang{中文}
\def \thistext{中文文本}
\def \olympiad{第六届国际理论, 数理及应用语言学奥林匹克竞赛}
\def \pgheader{第六届国际语言学奥林匹克竞赛}
\def \bulgaria{保加利亚}
\def \sunbeach{阳光海滩}
\def \olydates{2008年8月4 — 9日}
\def \probindl{个人赛题目}
\def \solsindl{个人赛解答}
\def \probteam{团体赛题目}
\def \soluteam{团体赛解答}
\def \probword #1{题 \##1}
\def \pontword{分}
\def \rulesmot{规则}
\def \regulats{解答规则}
\def \regulatx{毋需抄题. 将不同问题的解答分述于不同的答题纸上. 每张纸上注明题号、座位号和姓名}
\def \towarrant{否则答题纸可能被误放或遗失.}
\def \regulaty{解答需详细论证. 无解释之答案, 即便完全正确, 也会被处以低分.}
\def \editorsz{编者}
\def \edinchef{主编}
\def \goodluck{祝你好运}
\def \answersp{答案}
\def \fordinto #1{翻译成#1}
\def \outdrehu #1{翻译 #1}
\def \corrcorr{找出正确的对应关系}
\def \transcri{转写这些词}
\def \writorth{用 Listuguj 拼写法拼写}
\def \macschwa #1{在这个{\tscr}中 #1 $\approx$ 英语 \word{abbot} 中的 \word{o}}
\def \zoqwedge{$\approx$ 英语 \word{but} 中的 \word{u} }
\def \oshiroko{相当于 open~\word{o}} % open 怎么翻显得文艺一些呢。。
\def \eshiroko{$\approx$ 英语~\word{aspen} 中的 \word{a}}
\def \onororth #1#2#3{#1 $\approx$ 英语 \word{cat} 中的 \word{a}, #2 = 法语 \word{eu} 或德语 \word{ö} (这些字母表示长元音). #3 读作短的 #2}
\def \owumlaut{= 法语 \word{eu} 或德语 \word{ö}}
\def \chumlaut{= 法语 \word{eu} 和 \word{u} (德语 \word{ö} 和~\word{ü})}
\def \uyumlaut{= 法语 \word{u} 或德语 \word{ü}}
\def \nbsuperw{表示前面的辅音用圆唇发音}
\def \ruleforw #1#2{字母 #1 表示辅音后圆唇, 而 #2 则不是}
\def \rquvular #1#2{字母 #1 表示一个 `巴黎的' \word{r} (深喉处发音), 字母 #2 表示同样发音位置的一个如同 \word{k} 的发音}
\def \mandcons #1#2#3#4{#1 和~#2 与英语硬辅音 \word{sh} 中的 \word{shut} 和 \word{ch} 中的 \word{chuck}, #3 和~#4 相当于英语清辅音 \word{sheet} 中的 \word {sh} 和 \word{cheat} 中的 \word{ch}}
\def \zoqconsz #1#2{#1 $\approx$ \word{hands} 中的 \word{nds}, #2 = \word{sh}}
\def \cjmicmac #1#2{#1 = \word{church} 中的 \word{ch}, #2 = \word{judge} 中的 \word{j}}
\def \spunvoid{是特定的清辅音}
\def \velarnas{= \word{hang} 中的 \word{ng}}
\def \palatnas{$\approx$ \word{onion} 中的 \word{ni}}
\def \jotisjot{nay}
\def \jotsound{= \word{yay!} 中的 \word{y}}
\def \tsaffric{$\approx$ \word{hats} 中的 \word{ts} (作为单辅音发音)}
\def \frivelar{= 苏格兰语 \word{loch} 中的 \word{ch}}
\def \dittovoi{的发音没有变, 但是需要浊化}
\def \glotstop{是一种特定的辅音 (所谓的声门{\stp})}
\def \retrflex{$\approx$ \word{word} 和 \word{art} 中的 \word{d} 和 \word{t}, 发音时舌尖向后}
\def \thorneth #1#2#3{英语 #2 和 #3 中的 #1 }
\def \singsyll{作为单音节发音}
\def \hetteken{标记}
\def \aspirath{表示前面的{\stp}辅音需要送气 (发音时出气)}
\def \longmark{表示长元音}
\def \apostrof #1{撇号在元音后表示音长, 在辅音后读作 #1 }
\def \schwason #1{虽然没有写出, 但是在任何辅音与响音 #1 之间发音}
\def \schwabis{也在辅音连缀前的词首发音}
\def \rulestop #1#2#3{#1 在词首或元音之间读作\voicon{} (#2) ,  在词尾或临近另一个辅音读作清辅音 (#3) }
\def \incanada{在加拿大}
\def \northcan{在加拿大北部}
\def \inchiapa{在墨西哥南部的 Chiapas 省}
\def \spokenca #1#2{#2, 约有#1人使用该语言}
\def \drehname{Drehu 语}
\def \drehspok #1{在新喀里多尼亚东部的利富岛,有超过 #1 人使用 Drehu 语}
\def \indrehus{在 Drehu}
\def \cemuname{Cemuhî 语}
\def \cemuspok #1{在新喀里多尼亚的东部海岸约有 #1 人使用 Cemuhî 语}
\def \inukname{因纽特语}
\def \geninukt{因纽特语 (加拿大因纽特) 属于爱斯基摩-阿留申语系}
\def \zoquenom{Copainalá Zoque}
\def \genzoque #1{ #1 语属于 Mixe-Zoque 语系}
\def \bothaune{两种语言都属于南岛语系}
\def \gemicmac{米克玛克语是一种阿尔冈昆语}
\def \inmicmac{下面是以所谓的 Listuguj 拼写法拼写的米克玛克语的单词, 他们的语音{\tscr}及汉语翻译}
\def \indrecem{下面是两种新喀里多尼亚语言的单词和词组}
\def \giveword #1{下面是 #1 中的词语}
\def \inscalds #1{下面是四段于公元900年创作的古诺斯语诗的摘抄. 他们全部根据名为 #1 (lit.`court meter')的meter书写} % 这个meter到底是个神马。。
\def \insentin #1{下面是 #1 中的语句}
\def \redrecem #1#2{下面是几个从 #1 翻译成 #2 的词语}
\def \dqstanza{下面是一篇被抹去13个词语的诗节}
\def \theirtra{他们的汉语翻译}
\def \chaotict{(乱序)}
\def \cdaskone #1#2#3#4{你认为单词 #1 和 #3 分别在 #2 和 #4 中是什么意思}
\def \cdasktwo #1#2#3#4#5{在 #1 中 #2 就是 `#3' ,  #4 就是 `#5'}
\def \allitera{头韵}
\def \videstat{见题目的陈述部分}
\def \allitert #1{头韵是 #1 的一个主要原则}
\def \alliterh{每一联(两行)的第一行都包含两个互押头韵的词语, 并且第二行的首个单词也押相同的音}
\def \vocallit #1{所有的元音都互相并且和 #1 押韵}
\def \nunirule{但这并不是所有的规则}
\def \mdiffers{上面的文本都是从一个以上的手稿中摘抄下来的}
\def \mvariats{有时不同文本的相应位置会出现词语的不同变体, 而学者们必须从这些变体中判断哪些是原版}
\def \huchoose{不同的思考会推动结论的发现. 有时诗篇的规则会帮助判读哪些变体是错误的}
\def \nonlybut #1#2#3{例如, 在行 #1 我们不仅会找到 #2, 还会找到 #3}
\def \outother #1#2{就会根据诗的结构而被排除, 但是 #1 和 #2 都适合此行, 所以我们需要其他原因来推敲这些词语.}
\def \drotqtri #1#2#3{在行 #1 中 #2 和 #3 出现在手稿中, 但是 #3 并不满足诗的要求.}
\def \vadrules #1{描述你在 #1 中观察到的一联诗中的规则}
\def \dqfiller{下表中包含所有13个被抹去的词语(按字母顺序)和两个不属于诗节 V 的词语}
\def \fillgaps{填补诗节 V 中的空白}
\def \wwonorse{古诺斯语是一种北日耳曼语. 大约在公元700年至1000年有人使用过.} 
\def \onornorm{题目中所有诗歌的范例都以标准化的拼写给出, 并且符合所有题材上的规则}
\def \qtysylla{音节数. 每行有六个音节}
\def \inrhymes{中韵}
\def \inrhymep{将每行的元音 (和复合元音) 标示成}
\def \inrhymeq #1#2{至少有一个辅音直接接在 V$_5$ 后面也会直接接在 #1 #2 后面}
\def \inrhymer{并且, 偶数行}
\def \egdrotts #1{例如, cf.\ 行 #1 (头韵用粗体标出, 中韵用下划线标出)} % cf.\ 是什么东东呢。。。
\def \excesswd{剩余的词语}
\def \autelexp{圣所是一座教堂中最神圣主要的部分}
\def \autelmot{圣所}
\def \abeilles{一群蜜蜂}
\def \abeillex{蜜蜂}
\def \abeimult{蜜蜂}
\def \animalwd{动物}
\def \aquagran{水}
\def \banamult{香蕉}
\def \bananasz{一串香蕉}
\def \bananawd{香蕉}
\def \biblemot{圣经}
\def \boiremis{用来喝的}
\def \bonemult{骨头}
\def \boneword{骨头}
\def \calendar{日历}
\def \carrelet{锥子}
\def \costemot{岸}
\def \dimanche{星期天}
\def \dormimot{用来睡觉}
\def \ecclesia{教堂}
\def \esperonx{马刺}
\def \esperony{戳动物的工具}
\def \fourchet{叉子}
\def \frontier #1{#1的边界}
\def \holyhome{神圣的}
\def \holyjour{神圣的}
\def \holypost{神圣的}
\def \holytome{神圣的}
\def \homegran{房子}
\def \homeword{房子}
\def \jourmult{日期}
\def \jourword{日期}
\def \kurkalem{铅笔}
\def \litparol{床}
\def \multitud #1{大量的#1}
\def \murparol{墙}
\def \nuitgran{黑夜}
\def \piquemis{用来戳的}
\def \piquemot{用来戳的}
\def \tool #1{#1工具}
\def \postdorm{用来睡觉的地方}
\def \postword{地方}
\def \schrimis{用来写的}
\def \schrimot{用来写} 
\def \skeleton{骨架}
\def \tomeword{书}
\def \twilight{黄昏}
\def \verremot{杯子}
\def \shamanex{萨满在某些文化中可能是牧师, 术士或医师}
\def \thejager{猎人}
\def \deshaman{萨满}
\def \sinukoer{你的狗}
\def \inuktitq{你来了}
\def \inuktitk{你摔倒了}
\def \inuktitl{你伤到了你自己}
\def \shottest #1{你射杀#1}
\def \inuktitr{你治愈了猎人}
\def \inuktitc{你刺穿了狗}
\def \adogword{一只狗}
\def \tteacher{教师}
\def \inuktitj{一位医生}
\def \inuktitw{伤到了一位教师}
\def \inuktits{伤到了你}
\def \inuktitb{看到了你}
\def \inuktitg{刺穿了狼}
\def \inuktiti{射杀一只北极熊}
\def \inuktitf{男孩射杀了医生}
\def \inuktite{狗伤到了教师}
\def \inuktitm{狼看见了你的萨满}
\def \inuktitu{你的狼摔倒了}
\def \inuktith{北极熊来了}
\def \inuktitn{你的北极熊伤到了一个男孩}
\def \inuktita{医生治愈了你}
\def \inuktitd{你的医生治愈了你的男孩}
\def \inuktito{你的猎人治愈了他自己}
\def \inuktitt{教室看到了男孩}
\def \withword{`和'}
\def \forparol{为了}
\def \abovemot{上方}
\def \renderpl{复数}
\def \zadiente{为了牙(复数)}
\def \zacourge{为了壁球}
\def \courges{壁球(复数)}
\def \cucourge{和壁球}
\def \nakpat{一只仙人掌}
\def \kittel{一只水壶}
\def \zakittels{为了水壶(复数)}
\def \sukumguy{城镇上方}
\def \zakumguys{为了城镇(复数)}
\def \balkan{一座山}
\def \balkanes{山}
\def \cumokmok{和玉米}
\def \cumikittel{和我的水壶}
\def \mimokmok{我的玉米}
\def \mipillar{我的邮件}
\def \sumihimmel{我的天空上方}
\def \supillar{邮件上方}
\def \pillars{邮件(复数)}
\def \tudientes{你的牙}
\def \cutunakpat{和你的仙人掌(复数)}
\def \tukumguy{你的城镇}
\def \tucaycay{你的藤蔓}
\def \himmel{一片天空}
\def \poxkuy{一把椅子}
\def \zapoxkuy{为了椅子}
\def \supoxkuys{椅子(复数)上方}
\def \zacaycay{为了藤蔓}
\def \sucaycay{藤蔓上方}
\def \shadow{一片阴影}
\def \shadows{阴影(复数)}
\def \sushadows{阴影(复数)上方}
\def \cushadow{和阴影}
\def \axeparol{斧头}
\def \unsafewd{不安全}
\def \arxangel{天使长}
\def \shoeword{钉蹄铁 \emph{(给马)}}
\def \spoonmot{匙}
\def \agentmot{印第安代理人}
\def \lawparol{法律}
\def \lieontop{躺在顶部}
\def \indianfe{印第安女人}
\def \oheadmot{上方, 在头顶上}
\def \stovemot{炉}
\def \foolishn{愚蠢 (名词)}
\def \springwa{泉水}
\def \raspberr{树莓}
\def \worships{崇拜}
\def \goatword{山羊}
\def \southmot{南方}
\def \snakemot{蛇}
\def \lookroon{环顾四周}
\def \therefor{所以, 因此}
\def \defobj #1{明确的物体}
\def \cedreord{在这两种语言中, 修饰语后置}
\def \instruct{因纽特语具有下列基本结构}
\def \wherinuk #1#2{其中 #1 是名词,  #2 是动词}
\def \declinuk #1#2#3{如果一个名词在他是一个\defobj{}或一个没有\defobj{}的句子的主语时,词尾接 #1, 那么在他是一个不\defobj{}时, 就要在尾缀 #3 之前接 #2 }
\def \sayyours #1#2#3#4{要表示 `你的', #1 就要换成 #2, #3 换成 #4}
\def \zoquesfx{此题中的名词词缀有}
\def \suffinuk{动词接以下词缀}
\def \jortinuk #1#2{尾音是元音接 #1 , 是辅音接 #2 }
\def \endingin{根据主语或\defobj{}(如果存在的话)的人称附加尾缀}
\def \schemata{前两种结构中}
\def \schematr{第三种结构中}
\def \transref{一个没有宾语的及物动词当作反身动词处理}
\def \simnasal #1#2#3{ {\stp} #2 在鼻辅音 (#1) 后浊化 (分别是 #3 )}
\def \metathyk #1#2{如果 #1 接在 #2 后面, 两音互换位置}
\def \possezoq #1#2{此题中的物主代词有 #1 `我的' 和 #2 `你的'} % possessive pronouns 是物主代词吧。。
\def \prenasal{如果名词以{\stp}起始, 这个辅音浊化并且在前面加上相应的鼻音}
\def \hh{送气}
\def \stp{塞音}
\def \frc{擦音}
\def \tscr {转写}
\def \transcrn{\tscr}
\def \fqts #1{反切\tscr#1}
\def \risetone{升}
\def \flattone{平}
\def \falltone{降}
\def \altotone{高}
\def \bajotone{低}
\def \Guangyun{\textsl{广韵}}
\def \xarakter{汉字}
%\def \chinname{Chinese}
\def \mandname{普通话}
\def \cantname{广东话}
\def \putongww{普通话是中国的官方语言, 基于北京方言. 约八亿五千万人使用该语言}
\def \wuandyue{九千万人使用吴语 (上海话), 七千万人使用广东话 (粤语).}
\def \whencomp #1{在编译#1时}
\def \homochin #1#2{在编译#1字典时 (#2), 汉语相当同质化}
\def \nophonet{由于汉字不是表音文字}
\def \oneastwo{该字典使用了一套简单的系统, 通过两个汉字来表示一个汉字的发音, 而读者理应知道前者的发音 (他们是常用字)}
\def \calledfq{这套系统叫做\emph{反切}.}
\def \hetechin{后来, 虽然汉语方言分化, 但许多古代的\fqts{}仍可使用, 只不过在不同的方言中有不同 (且更复杂) 的使用方法.}
\def \herecant{下面是一些反切{\tscr}. 每个汉字的广东话读音亦给出.}
\def \heremand{下面是另外一些{\tscr}, 但只给出了其普通话读音}
\def \canmando{下面是些汉字及其广东话和普通话的读音}
\def \canmanre{下面是另外一些汉字及其广东话和普通话的读音}
\def \hucanton{解释古代\fqts{}是如何应用于现代广东话的}
\def \huwasuse #1{#1\fqts{}是如何工作的}
\def \whatonly #1{在广东话中, 上述{\tscr}只有一个可以应用这条简单的老规则来得到正确的结果. 哪一个呢?}
\def \nownovoi #1{在多数当代汉语方言中 (包括广东话和普通话) 不存在\voicon{}, 除了响音 (#1)}
\def \ccvoiced #1{在#1编译时汉语存在其他\voicon{}, 他们后来并成对应的清音}
\def \fromvoid{浊}
\def \voicon #1{浊辅音}
\def \tounvoid{清}
\def \devofric #1#2{#1{\frc}变成#2{\frc}}
\def \devostop #1#2{#1{\stp}变成{\hh}或不{\hh}#2{\stp}}
\def \wuvoiced{浊音在吴语中得到了保留}
\def \wuvoicex #1#2#3#4{比如, 汉字 #1 在吴语中发作 #2, 在广东话中发作 #3, 在普通话中发作 #4}
\def \wherevoi #1{上一节中的哪些汉字#1首辅音为浊辅音}
\def \whyvdasp{这些\voicon{}在广东话中变得{\hh}或者不{\hh}取决于什么条件?}
\def \tredecar{在古汉语中存在四种声调, 但是只有三种出现在此题中}
\def \hutresex{解释这三种音调是如何演变出广东话的六种音调.}
\def \formanda{暂时忽略音调, 给出在普通话中使用古代\fqts{}的规则.}
\def \tonemiss{一些音调被移除}
\def \devemand{描述音调和浊首辅音是如何在普通话中演变的}
\def \humandar{可以总结出哪些在普通话中读\fqts{}的音调的规则?}
\def \rarewhat{一些初始的辅音和音调组合在现代普通话中极为罕见. 哪些呢?}
\def \supptone{判断出遗失的音调是哪些}
\def \readfoll #1{给出下面的\tscr#1的读法}
\def \incanton{在广东话中}
\def \inmandar{在普通话中}
\def \infohave{一些{\tscr}本不可读, 但这道题包含了足以读出他们的信息}
\def \triparts{汉语音节由三部分组成}
\def \onrimton{声母 (首辅音, 可能不存在如 3B), 韵母 (后面的所有音) 和声调}
\def \cantones #1#2{广东话音调可以认为存在两种不同的性质: 音高 (#1) 和轮廓 (#2).}
\def \socanton{若要在广东话中使用\fqts{}, A 的声母和声调音高将与 B 的韵母及声调轮廓组合} % 错别字:“与”/“于”
\def \decanton #1#2#3{但是如果 A 的 (和 X 的) 声调是#1, X的声母(如果是一个\stp), 必须在 B 的 (和 X 的) 声调是#2时{\hh}, #3时不{\hh}}
\def \onsarhyb{显然声母来自 A 字, 韵母来自 B 字}
\def \aspirodd{但是送气的规则却很奇特}
\def \noterken{可能他并不属于原本的反切系统}
\def \tonfromb{而声调是不是可能只来自于两个字中的一个呢? 那就只能是 B 了, 因为旧的规则只能在一个{\tscr}中给出正确结果.}
\def \urfanqie{因此原本的反切规则是: A 的声母与 B 的韵母及声调组合}
\def \onlyelve{根据这个规则, 只有{\tscr}11能被读出来}
\def \sonorlow #1{根据那些带有一个响音声母的音节, 我们可以看出他们一直都是低音调 (#1)}
\def \voitolow{假设所有广东话中的\voicon{}都以相似的方式演变, 我们就可以得出哪些在之前有浊辅音声母的现在是低音调}
\def \volosupp{这些对于吴语例子中的字来说也是正确的. \textbf{(d)}中的文本支持这个观点.}
\def \sherevoi{因此声母是浊辅音的字有}
\def \ifiatall{如果他有一个声母他}
\def \voidevel #1#2{浊化{\stp}在声调是#1时{\hh}, #2时不{\hh}}
\def \contoura{广东语的音调轮廓符合文言的三个声调}
\def \nuheight{音调高度是由\voicon{}的演变所带来的创新.}
\def \explaict #1{因此我们可以解释为什么\fqts{}#1这样读}
\def \samehigh{ X 字与 A 有相同的音高, 因为他的声母来自于 A, 而且广东话的音高是由文言声母的发音决定的}
\def \otherasp{但是如果声母是一个浊化的\stp, 他在 X 和 A 中就会有不同的演变方式, 因为他的送气是由音调轮廓决定的, 也就是 X 从 B 中得到的, 并且可能会和 A 的轮廓不同.}
\def \mandcomp{在普通话中, 声母和韵母并不是像广东话这样直接组合在一起的}
\def \conoccur #1#2#3{可以注意到的是在 #1 的后面我们总会发现 #2, 而 #3 却从来不接这两个元音}
\def \dejaknow{我们已经知道声母来自 A, 韵母来自 B }
\def \whenrest{当应用上述规则时}
\def \ruharden #1#2#3#4{#1 就会消失,  #2 在 #4 后面就会变成 #3}
\def \rusoften #1#2#3{#1 在 #3 前变成了 #2}
\def \usethese #1{#1使用\fqts{}同样需要应用如上规则}
\def \onsetcis #1#2{如果 #1 的声母是 #2 并且}
\def \rhymebei #1#2{B 的韵母既不是以 #1 起始, 也不是以 #2 起始}
\def \onsetane{A 的声母也不是以上任何一个}
\def \indeterx #1{我们不能决定 X 的 #1 时什么}
\def \acconset{声母}
\def \accrhyme{韵母}
\def \vocanton{根据广东话音节的音调,  我们可以判断文言中的声母是否浊化}
\def \mandtone{普通话中音调的变化如下}
\def \mandtons #1#2#3{#2 在声母是 #1 的情况下, #3 除外}
\def \voinoson{浊辅音但不是响音}
\def \unvoiced{清辅音}
\def \decontur{我们可以看到这里并不保留轮廓}
\def \somanton #1{#1的声调在\fqts{}中读音如下}
\def \legecons #1#2#3#4{其中 #1 表示一个响音, #2 表示一个 \frc, #3 表示不{\hh}, 而 #4 表示一个{\hh}的\stp}
\def \dimanton{因此大部分情况下 X 在普通话中的音调不能从 A 的音调或 B 的音调中模棱两可地得出, 但是有一些特殊情况}
\def \rarethat{带有一个响音声母并在声调5或带有一个不{\hh}声母并在声调35上的音节不可以存在于普通话中 (如果存在, 规则中就要加入特殊情况了).}
\def \chintons{每个汉语方言都有固定数目的音调 (每个音节发音的旋律)}
\def \chaotons{本题使用了语言学家赵元任提出的系统, 其用数字 1 (最低) 到 5 (最高) 来标记音高的五级, 并将音调转写成连续的音级}
\def \havetons{你所需的全部音调都在本题中出现了}
\def \zidenote{如果你不想写汉字, 你可以使用{\tscr}的序号并指明具体的汉字来指代他们: X (被转写的)}
\def \nrintscr #1{#1用于{\tscr}的汉字}
\def \premiere{第一个}
\def \deuxieme{第二个}
\def \nnovowel #1{注意普通话的 #1 汉字的读法不包含元音}
\def \enquetea{你做了哪些题目?}
\def \enqueteb{你最喜欢哪道题?}
\def \enquetec{你觉得哪道题最难?}
\def \enqueted{你觉得哪道题最简单?}
\def \keredomo{然而,}
\def \alwaysso{全部都是}
\def \eg{例如,}
\def \et{和}
\def \au{或}
\def \Vhimself{\textsf{V} (他自己)}
\def \aYrender{一个 \textsf{Y}} % a 就翻译成“一个”可以么。。
\def \theYrend{\textsf{Y}}
\def \solo{仅仅}
\def \just{就}
\def \rite{恰好}
\def \como{像}
\def \cosi{好像}
\def \ABname{Alexander Berdichevsky}
\def \BBname{博日达尔·博扎诺夫}			% Bozhidar Bozhanov
\def \SBname{Svetlana Burlak}
\def \DGname{德米特里·格拉西莫夫}			% Dmitry Gerasimov
\def \XGname{克谢尼娅·吉利亚洛娃}			% Ksenia Gilyarova
\def \IGname{Ivaylo Grozdev}
\def \SGname{斯坦尼斯拉夫·古列维奇}		% Stanislav Gurevich
\def \IDname{戴谊凡 (伊万·德尔然斯基)}		% Ivan Derzhanski
\def \AHname{Adam Hesterberg}
\def \BIname{鲍里斯·伊奥姆丁}				% Boris Iomdin
\def \IIname{Ilya Itkin}
\def \RPname{Renate Pajusalu}
\def \APname{亚历山大·皮佩尔斯基}			% Alexander Piperski
\def \MRname{玛丽娅·鲁宾斯坦}				% Maria Rubinstein
\def \LFname{Ludmilla Fedorova}
\def \TTname{Todor Tchervenkov}
\def \LMSname{刘闽晟}
\def \LHname{刘晗} 
\def \edinames{\ABname, \BBname, \SBname, \IDname\ (\edinchef), \LFname, \DGname, \XGname, \IGname, \SGname, \AHname, \BIname, \IIname, \RPname, \APname, \MRname, \TTname}
\def \whowroti{\LMSname, \LHname}
\def \whowrotj{\LMSname, \LHname}

% READ ME!

% 关于翻译中的问题:	
% 某些语言名称的中文翻译我查不到;
% voiced和voiceless都是清(辅)音么;
% contour翻译成“起伏”怎么样?
% 所以的“它”都被翻译成了“他”。
% 对于某些好长(也可能不是很长)的句子,我可能翻译的不太通顺
% 细节上还需调整润色,可能有专家前后浏览一遍就能看出来是不同的人翻译的(假设他不看我们的名字)
% 题我没做,,LMS的翻译我只是粗略的浏览了一下
% 我书读的少,,不要骂我。。。

