\usepackage[british]{babel}
\emergencystretch=20pt
\usepackage{a4wide,euler}
\usepackage{fontspec}
\usepackage{xeCJK}
\usepackage{xeCJKfntef}
\usepackage{fontspec}
\usepackage{xeCJK}
\usepackage{ifplatform}

% Load Computer Modern Unicode.
% As this file will be copied to work path, we use ../../../ here.
\setmainfont[
  Path = ../../../fonts/CMUnicode/,
  Extension = .ttf,
  BoldFont = *-Bold,
  ItalicFont = *-Italic,
  BoldItalicFont = *-BoldItalic,
  SlantedFont = *-RomanSlanted,
  UprightFont = *-Roman
]{CMUSerif}
\setsansfont[
  Path = ../../../fonts/CMUnicode/,
  Extension = .ttf,
  BoldFont = *-Bold,
  ItalicFont = *-Oblique,
  BoldItalicFont = *-BoldOblique,
  UprightFont = *
]{CMUSansSerif}

% Load Chinese Fonts.
% Due to the license beetween SinoType and Apple, we couldn't use SinoType's fonts under Windows or Linux.
% If non-Mac users contribute to our project, they should add proper fonts by themselves.
% Note that "-shell-escape" should be used for XeLaTeX. For details, check out ifplatform's documentation.

\ifmacosx
\setCJKmainfont[
  BoldFont = Songti SC Bold,
  ItalicFont = Kaiti SC Regular,
  BoldItalicFont = Kaiti SC Bold,
  SlantedFont = STFangsong
]{Songti SC Regular}
\setCJKsansfont[
  BoldFont = Heiti SC Medium,
  ItalicFont = Kaiti SC Regular,
  BoldItalicFont = Kaiti SC Bold
]{Heiti SC Light}
\else
\fi

% Use Computer Modern Serif for those characters.
\normalspacedchars{•‘’“”1234567890-—–/̄́`ˇ}
\DeclareFontFamily \encodingdefault {ipa}{}
\DeclareFontShape \encodingdefault {ipa}{m}{it} {<-> ipait10 }{}
\newcommand \ipa {\fontfamily {ipa}\selectfont }
\newcommand \wipa [1]{\textit {\ipa #1}}
\newcommand \word [1]{\textit {#1}}
\newcommand \egar [1]{\mbox{\ipa \itshape #1}}
\newcommand \by [1]{%
  \unskip\nobreak\hfil\penalty50
  \hbox{}\nobreak\hfill{\hbox{\qquad#1}}\par}
\newcommand \A {\accent22a}
\newcommand \E {\accent22e}
\renewcommand \schwa {\accent127a}
%\newcommand \schwa {@}
\newcommand \sh {\accent20s}
\newcommand \sj {\accent19s}
\newcommand \cj {\accent19c}
\def \<#1>{\leavevmode\hbox{$\langle$#1\/$\rangle$}}

\newcounter{items}
\newenvironment{items}{
  \begin{list}
  {(\alph{items})}
  {\usecounter{items}}}
 {\end{list}}

\font \tenta=linz10
\font \seventa=linz7
\newfam \tafam
\newcount \tafamcount
\tafamcount=\tafam \multiply\tafamcount by 256
\def\ta{\fam\tafam\tenta}
\textfont\tafam=\tenta
\scriptfont\tafam=\seventa

\newcommand \It {\dot I}
\newcommand \He {\dot \Lambda}
\newcommand \She {\dot \Delta}
\newcommand \she {\dot {\ta d}}
\newcommand \famover [1]{\frac{\He\She i\she}{#1}}

\newcommand \heart {{\ta H}}
\newcommand \work {{\ta W}}
\newcommand \talk {{\ta T}}
\newcommand \eat {{\ta B}}
\newcommand \scribe {{\ta P}}
\newcommand \epistle {{\ta L}}
\newcommand \orphform {$\frac{i\she(-\He\She)}{(-\He\She)}$ `\kidsword\ ($-$\pareword)/($-$\pareword)}

\newcounter {rowcount}
\newcommand \birow [1]{\stepcounter {rowcount}\therowcount. & #1 \\}
\newcommand \trirow [2]{\stepcounter {rowcount}\therowcount. & $#1$ & #2 \\}

\newcommand \biline [3]{\word{#1#2} & #3 & \word{#2} & #3 \\}

\makeatletter
\newcommand \procherk {\multicolumn{2}{l}{\leaders \hbox {\textbar }\hfill \kern\z@ }}
\def \ps@somestyle {
      \let\@oddfoot\@empty
      \let\@evenfoot\@empty
      \def\@oddhead{\hfill \textsl{\pgheader\quad\chapname }}
      \def\@evenhead{\hfill \textsl{\pgheader\quad\chapname }}}
\newcommand \theassignment {\@arabic\c@assignment}
\makeatother

\newcounter {assignment}[section]
\def \toCmnNum #1{\ifcase#1\or 一\or 二\or 三\or 四\or 五\fi}
\newcommand \problem [1]{\section*{第\stepcounter {section}\toCmnNum{\thesection}\probword\ (#1 \pontword)}}
\newcommand \solution {\section*{\soluword\ \stepcounter {section}\thesection}}
\newcommand \assignment {\stepcounter {assignment}\paragraph{\asgtword\toCmnNum{\theassignment}:}}

\pagestyle{somestyle}

\newcommand \makepart [1]{\newpage
\begin{center}%
{\LARGE \olympiad \par }
\vskip 1em{\begin{tabular}[t]{c}\Large \whenwher
\end{tabular}\par }
\vskip 1em{\large #1}\end{center}\par \vskip .5em
\def \chapname {#1}\setcounter {section}0}

%%% bgn shortcut defs

\def \linzdata #1#2#3#4#5#6#7#8{\begin{tabular}{rll}
\trirow {(\famover{\She i\she}+\frac{i\she}\she)^\talk}{\smallskip #1。}
\trirow {n(>\It)^{\work -t}}{\smallskip #2。}
\trirow {(\frac{i\she(-\He\She)}{(-\He\She)})^\scribe=\epistle}{\smallskip #3。}
\trirow {(-n\It_1)^\scribe-t=\It_2}{\smallskip #4。}
\trirow {\epistle^{\sqrt \scribe}-t=-\She_3}{\smallskip #5。}
\trirow {(\famover{\She i\she})^{-\heart}=\work}{\smallskip #6。}
\trirow {((>\It)-\heart)^\eat-t=\famover{i\she}}{\smallskip #7。}
\trirow {\She_3^{-t}}{#8。}
\end{tabular}}

\def \linztest {\begin{tabular}{rl}
\birow {\smallskip $\It_3^{\heart-\sqrt\heart}$}
\birow {\smallskip $(\famover{\He\She i}-\talk)^\scribe+t
=\famover{\She i\she}+\famover{\He i\she}$}
\birow {\smallskip $\She_2^{\work+t-\talk}-t$}
\birow {$\epistle^{\sqrt \eat}-t=\frac{i\she}i-\eat$}
\end{tabular}}

\def \linzglos #1{$\He$ `\maleword', $\She$ `\femaword', $i$ `\klukword',
$\she$ `\girlword', $\epistle$ `\lettword', $\work$ `#1'.}
\def \linzglot #1#2{$\He\She$ `\maleword\ + \femaword\ = #1',
$i\she$ `\klukword\ + \girlword\ = #2',
$\He\She i\she$ `\maleword\ + \femaword\ + \klukword\ + \girlword\ = \famiword'.}
\def \linzglou #1#2{$\famover{\She i\she}$ `\famiword/(\femaword\ + \kidsword) = #1',
$\frac{i\she}{\she}$ `\kidsword/\girlword\ = #2',
$\famover{i\she}$ `\famiword/\kidsword\ = \pareword'.}

\def \linzdone {\begin{tabular}[t]{rl}
\birow{\smallskip $(\She_1+\frac{\He\She}\She)^\talk-t=-n\It_3$}
\birow{\smallskip $(n\It)^{{\work}-\heart}$}
\birow{\smallskip $(\frac{\She(-\He)}{(-\He)}+\heart)^\heart=(<\It)-{\work}$}
\birow{$(n\It_2)^{\sqrt \talk}+t$}
\end{tabular}}

\def \fracdata {\begin{eqnarray}
\egar{tumn} + \egar{tumn\E n} & = & \egar{talatt itm\A n} \\
\egar{sabaCt itl\A t} + \egar{suds} & = & \egar{Ca{\sh}art irb\A C} \\
\egar{tusC\E n} + \egar{tusC} & = & \egar{suds\E n} \\
\egar{xamast ixm\A s} + \egar{subC} & = & \egar{tamant isb\A C} \\
\egar{subC\E n} + \egar{xums\E n} & = & \frac{24}{35}
\end{eqnarray}}

\def \fracdone {%
(1) $\frac18 + \frac28 = \frac38$,
(2) $\frac73 + \frac16 = \frac{10}4$,
(3) $\frac29 + \frac19 = \frac26$,
(4) $\frac55 + \frac17 = \frac87$,
(5) $\frac27 + \frac25 = \frac{24}{35}$.}

\def \fractest {$\egar{rubC} + \egar{Ca{\sh}art its\A C} \; = \; \egar{sabaCt isd\A s}$}

\def \adygdata #1#2#3#4#5#6#7{\begin{tabular}{rll}
\trirow {\wipa{{\sj}any{\cj}yr hakum deR0@uco.}}{#1。}
\trirow {\wipa{syda laR0@m tyriz8@r@r?}}{#2?}
\trirow {\wipa{ax0{\sj}@r px0wantym tyreR0af@.}}{#3。}
\trirow {\wipa{{\sj}ywanyr Xanym tyreR0@uco.}}{#4。}
\trirow {\wipa{syda px0@n\d t@\d kum \d{{\cj}}iR0af@r@r?}}{#5?}
\trirow {\wipa{laR0@r tyda zy\d{\cj}iR0@ucor@r?}}{#6?}
\trirow {\wipa{laR0@r tyda zytyriz8@r@r?}}{#7?}
\end{tabular}}

\def \adygtest {\begin{tabular}{rl}
\birow{\wipa{px0@n\d t@\d kur hakum dez8@.}}
\birow{\wipa{ax0{\sj}@r tyda zydiR0af@r@r?}}
\end{tabular}}

\def \adygcons {\wipa{\cj}, \wipa{\d{\cj}}, \wipa{\d k}, \wipa{R0},
\wipa{\sj}, \wipa{\d t}, \wipa{x0}, \wipa{z8}, \wipa X}

\def \adyganal #1#2#3{\begin{tabular}{r|llr@{}r@{}r@{}l@{}l|r@{ }l@{ }l}
(1, 3, 4) & $X$-\wipa r & $Y$-\wipa m && $P$- & \wipa e- & $V$ &.& `#1.' \\
(2, 5) & \wipa{syda} & $Y$-\wipa m && $P$- & \wipa i- & $V$ & \wipa{-r@r}? & `#2?' \\
(6, 7) & $X$-\wipa r & \wipa{tyda} & \wipa{zy-} & $P$- & \wipa i- & $V$ & \wipa{-r@r}? & `#3?' \\
\end{tabular}}

\def \adygdone {\begin{tabular}[t]{rl}
\birow{\wipa{laR0@r {\sj}any{\cj}ym \d{\cj}eR0@uco.}}
\birow{\wipa{syda px0wantym \d{\cj}iz8@r@r?}}
\birow{\wipa{syda {\sj}ywanym diR0af@r@r?}}
\end{tabular}}

\def \adygmore #1#2#3{\begin{tabular}[t]{ll@{ }rr}
13.& \wipa{Xanyr tyda} & \wipa{zydiR0@ucor@r?} & #1? \\
13$'$.& \wipa{Xanyr tyda} & \wipa{zytyriR0@ucor@r?} & #2? \\
13$''$.& \wipa{Xanyr tyda} & \wipa{zy\d{\cj}iR0@ucor@r?} & #3? \\
\end{tabular}}

\def \frendone {\begin{tabular}{llll}
\assortiR {r\'e} \cureR {r\'e} \formeR {r\'e}
\former {re} \futer {r\'e} \lancer {re}
\munEreR {r\'e} \partiR {r\'e}
\end{tabular}}

\def \basqdata {\begin{tabular}{l@{\qquad}l}
\word{urtarrilaren hogeita hirugarrena, larunbata}; &
\word{abenduaren azken astea}; \\
\word{otsailaren lehenengo osteguna}; &
\word{ekainaren bederatzigarrena, igandea}; \\
\word{abenduaren lehena, \underline{\qquad\qquad}}; &
\word{irailaren azken asteazkena}; \\
\word{azaroaren hirugarren ostirala}; &
\word{urriaren azken larunbata}; \\
\word{irailaren lehena, astelehena}; &
\word{\underline{\qquad\qquad} bigarrena, ostirala}.
\end{tabular}}

\def \basqmore #1#2#3#4{\begin{tabular}[t]{ll}
 \word{#1} & \word{abenduaren lehenengo astelehena} \\
 \word{#2} & \word{azaroaren hogeita bederatzigarrena, larunbata} \\
 \word{#3} & \word{urtarrilaren bigarren astea} \\
 \word{#4} & \word{otsailaren hirugarrena, astelehena} \\
\end{tabular}}

\def \tochtest {\begin{quote}
\wipa{\d st\A n5k}, \wipa{walo}, \wipa{r{\schwa}skare}, \wipa{\A sar},
\wipa{astare}, \wipa{\A \d st{\schwa}r}, \wipa{\A stre}, \wipa{as\A re},
\wipa{st\A n5k}, \wipa{w{\schwa}l}, \wipa{wlo}, \wipa{prats\A ko}, \wipa{pratsak},
\wipa{\A knats}, \wipa{akn\A tsa}, \wipa{tsan5k{\schwa}r}, \wipa{ts{\schwa}n5k{\schwa}r},
\wipa{kramartse}, \wipa{kr\A m{\schwa}rts}, \wipa{r{\schwa}sk{\schwa}r}, \wipa{sam}, \wipa{s\A m},
\wipa{ys\A r}, \wipa{s\A k{\schwa}r}, \wipa{yasar}, \wipa{s\A kre}, \wipa{ys\A r}.
\end{quote}}

\def \tochmore #1#2#3{\begin{items}
\item \wipa{st\A m}, \wipa{\d st\A m} `#1';
\item \wipa{rt{\schwa}r}, \wipa{ratre} `#2';
\item \wipa{p{\schwa}rs}, \wipa{parso} `#3'.
\end{items}}

\def \tochdone {\begin{tabular}[t]{ll|ll|ll}
A & B & A & B & A & B \\ \hline
\wipa{\d st{\A}n5k} & \wipa{st{\A}n5k}
& \wipa{{\A}knats} & \wipa{akn{\A}tsa}
& \wipa{pratsak} & \wipa{prats{\A}ko} \\
\wipa{{\A}\d st{\schwa}r} & \wipa{astare}, \wipa{{\A}stre}
& \wipa{kr{\A}m{\schwa}rts} & \wipa{kramartse}
& \wipa{r{\schwa}sk{\schwa}r} & \wipa{r{\schwa}skare} \\
\wipa{w{\schwa}l} & \wipa{walo}, \wipa{wlo}
& \wipa{s{\A}k{\schwa}r} & \wipa{s{\A}kre}
& \wipa{sam} & \wipa{s{\A}m} \\
\wipa{{\A}sar} & \wipa{as{\A}re}
& \wipa{ts{\schwa}n5k{\schwa}r} & \wipa{tsan5k{\schwa}r}
& \wipa{ys{\A}r} & \wipa{ys{\A}r}, \wipa{yasar} \\
\end{tabular}}

%%% end shortcut defs


\def \editrans {\vfill
\begin{flushright}
编辑:戴谊凡(主编),鲍里斯·伊奥姆丁,玛丽亚·鲁宾斯坦。\\
翻译:陈润,刘闽晟。
\end{flushright}}

%%% bgn language defs
%
\def \olympiad {第一届国际理论、数理及应用语言学奥林匹克竞赛}
\def \whenwher {保加利亚,波罗维茨,2003年9月8日 — 12日}

\def \pgheader {第一届 IOL:波罗维茨 '03}
\def \pontword {分}
\def \probword {题}
\def \soluword {题}
\def \asgtword {任务}
\def \fordword #1{翻译成#1}
%
\def \famiword {family}
\def \maleword {man}
\def \femaword {woman}
\def \klukword {boy}
\def \girlword {girl}
\def \pareword {parents}
\def \kidsword {kids}
\def \lettword {letter}
%
\def \assortiR #1{\word{#1assortir} & 再次采摘 & \word{assortir} & 采摘 \\}
\def \cureR #1{\biline {#1}{curer}{清洁}}
\def \formeR #1{\word{#1former} & 改革 & \procherk \\}
\def \former #1{\word{#1former} & 再次形成 & \word{former} & 形成 \\}
\def \futer #1{\word{#1futer} & 反驳 & \procherk \\}
\def \lancer #1{\word{#1lancer} & 再次投 & \word{lancer} & 投 \\}
\def \munEreR #1{\word{#1mun\'erer} & 赔偿 & \procherk \\}
\def \partiR #1{\word{#1partir} & 分配 & \procherk \\}
%
\def \comment {\paragraph {注:}}
%
\makeatletter
\def \@alph #1{\ifcase#1\or a\or b\or c\else\@ctrerr\fi}
\makeatother
%
%%% end language defs

\begin{document}
\makepart{个人赛}

\problem {20}
%
1916年,俄国学者雅各·林茨巴赫发明了一种世界通用的书写系统,他认为所有人不论何种母语都应该能够理解。林茨巴赫称他的新语言为「超越代数」。

以下是一些林茨巴赫的语言写的句子和中文翻译:

\medskip \linzdata
{[这]父亲和[这]兄弟 (单数) 在谈论}
{[这些]巨人们在不匆忙地工作}
{[这些]孤儿们在写一封信}
{不是我们写了你}
{[这]信不是她写的}
{[这]父亲不喜欢[这]工作}
{[这]邪恶的巨人吃了[这对]父母亲}
{她没有在匆忙}

\assignment \fordword {汉语}:

\medskip \linztest


\assignment 用「超越代数」表示:

\medskip
\begin{tabular}{rl}
\birow {我的丈夫和我 (即我和[这]丈夫) 没有谈论 (过去时) 他们。}
\birow {[这些]人们在不情愿地工作。}
\birow {[这]善良的寡妇喜欢[这]没工作的侏儒。}
\birow {你们将会被谈论。}
\end{tabular}
\medskip \\
%
请解释你的答案。
\by{(克谢尼娅·吉利亚洛娃)}

\newpage
\problem {25}
%
以下是埃及阿拉伯语%
\footnote{阿拉伯语的埃及方言有大约四千五百万人使用。由于埃及显著的经济、政治和文化影响,
更重要的是广播和电视节目的巨大数目与普及,说阿拉伯语其他方言的人普遍能够理解该方言。}
的数学等式。
%
所有的加数与和,除了最后一个,都由分子分母都不大于~$10$且分母不为~$1$的分数表示:
%
\fracdata
%
\assignment 写出数字形式的等式。
\assignment 等式\quad\fractest\quad 少了一个符号。补上该符号。
\comment
字母 \wipa{\sh} 发音为英语的 \word{sh},\wipa x 为单词 \word{loch} 中的 \word{ch};
\wipa C 是阿拉伯语特有的辅音。元音上的横线表示长音。

\by{(戴谊凡)}

\problem {15}
%
下列是一组巴斯克语%
\footnote{在巴斯克 (西班牙的一个自治区) 以及法国境内,有超过五十万人使用巴斯克语。巴斯克语目前未被证明与任何其他语言相关。}
%
及其乱序的汉语翻译 (一些词没有给出):

\medskip\basqdata
%
\begin{quote}
二月的第一个周四;
\underline{\qquad\qquad}的最后一个周三;
十二月一日,周三;十二月的最后\underline{\qquad\qquad};
六月九日,周日;一月二十三日,\underline{\qquad\qquad};
十月的最后一个周六;
十一月的第三个周五;
九月\underline{\qquad\qquad},周一;
一月二日,周五。
\end{quote}

\assignment
将原文与译文正确对应起来并填空。

\assignment \fordword {巴斯克语}:
%
\begin{quote}
十二月的第一个周一;
十一月二十九日,周六;
一月的第二周;
二月三日,周一。
\end{quote}
%
\assignment
你认为巴斯克语中星期里的日子 \word{astelehena\-}、\word{asteazkena} 和 \word{asteartea}字面翻译可能是怎样的?
\by{(亚历山大·阿尔希波夫)}

\newpage
\problem {20}
%
以下是一些使用一简化的拉丁化方案拼写的阿迪格语
\footnote{阿迪格语是属于阿布哈兹-阿迪格 (西北高加索语系) 语族。超过三十万人使用该语言,主要在阿迪格共和国 (俄罗斯联邦)。}
句子及其汉语翻译:

\medskip
\setcounter{rowcount}0
\adygdata
{他把[这]水壶放在[这]炉子里面}
{他把什么投在[这]碟子上面}
{他把[这]钱落在[这]箱子上面}
{他把[这]大锅放在[这]桌子上面}
{他把什么落在[这]凳子下面}
{他把[这]碟子放在哪里}
{他把[这]碟子投在哪里}

\assignment 请给出六、七两句更加精确的翻译。

\assignment \fordword {汉语}:

\medskip
\adygtest

\assignment \fordword {阿迪格语}:

\medskip
\begin{tabular}{rl}
\birow{他把[这]碟子放在[这]水壶下面。}
\birow{他把什么投在[这]箱子下面?}
\birow{他把什么落在[这]大锅里面?}
\end{tabular}

\assignment \fordword {阿迪格语},写出所有译法:

\medskip
\begin{tabular}{rl}
\birow{他把[这]桌子放在哪里?}
\end{tabular}

\comment
\adygcons\ 是阿迪格语特有的辅音, \wipa@ 和 \wipa y 是阿迪格语的元音。
\by{(Yakov Testelets)}

\newpage

\problem {20}
%
下列的表格包括了法语含前缀动词和对应的无前缀动词,以及所有的中文解释。
阴影区域表示含前缀动词没有对应的无前缀动词。一些动词的前缀未给出。

\medskip
\begin{tabular}{llll}
\word{r\'eagir} & 反应 & \procherk \\
\assortiR {\underline{\quad}}
\word{recommencer} & 重新开始 & \word{commencer} & 开始 \\
\word{recomposer} & 重新编写 & \word{composer} & 编写 \\
\biline {r\'e}{concilier}{和解}
\biline {r\'e}{conforter}{安慰}
\word{recr\'eer} & 再创造 & \word{cr\'eer} & 创造 \\
\word{r\'ecr\'eer} & 娱乐 & \procherk \\
\cureR {\underline{\quad}}
\word{redire} & 再说一遍 & \word{dire} & 说 \\
\word{r\'eduire} & 减少 & \procherk \\
\word{r\'e\'editer} & 再版 & \word{\'editer} & 出版 \\
\word{refaire} & 重做 & \word{faire} & 做 \\
\formeR {\underline{\quad}}
\former {\underline{\quad}}
\futer {\underline{\quad}}
\word{r\'eincarner} & 转生 & \word{incarner} & 具体化 \\
\word{rejouer} & 继续玩 & \word{jouer} & 玩 \\
\lancer {\underline{\quad}}
\munEreR {\underline{\quad}}
\word{r\'enover} & 翻新 & \procherk \\
\word{r\'eop\'erer} & 再次操作 & \word{op\'erer} & 操作 \\
\word{repartir} & 再次离开 & \word{partir} & 离开 \\
\partiR {\underline{\quad}}
\word{r\'ep\'eter} & 重复 & \procherk \\
\biline {r\'e}{sonner}{发声}
\word{r\'ev\'eler} & 揭露 & \procherk \\
\end{tabular}
\medskip \\
%
\paragraph{任务:}
利用以上表格的信息填空。解释你的答案。
\by{(鲍里斯·伊奥姆丁)}

\editrans
\makepart{Team Contest}

\problem {35}
%
In the first millennium CE there were in Chinese Turkestan
two closely related languages, Tocharian A and Tocharian B,
which had descended from a common ancestor, Proto-Tocharian.
Here are some Proto-Tocharian words
as they have been reconstructed by scholars:

\medskip
\begin{tabular}{|ll||ll||ll|}
\wipa{\A k{\schwa}natsa} & `unreasonable' & \wipa{p{\schwa}ratsako} & `chest (breast)' & \wipa{st\A n5k{\schwa}} & `palace' \\
\wipa{\A sare} & `dry' & \wipa{r{\schwa}s{\schwa}k{\schwa}re} & `sharp' & \wipa{ts{\schwa}n5k{\schwa}r} & `top' \\
\wipa{\A st{\schwa}re} & `pure' & \wipa{sam{\schwa}} & `same' & \wipa{w{\schwa}lo} & `king' \\
\wipa{k{\schwa}r\A m{\schwa}rtse} & `black' & \wipa{s\A k{\schwa}re} & `happy' & \wipa{y{\schwa}s\A r} & `blood' \\
\end{tabular}
\medskip \\
%
And here are Tocharian A and Tocharian B words
which are descendants of the Proto-Tocharian words listed above (in no particular order):

\tochtest
%
\assignment
Determine which word belongs to which language, knowing that:
%
\begin{itemize}
\item in one of the languages some words have two variants;
\item the first word is Tocharian A.
\end{itemize}

\assignment
Allocate the following words to languages and reconstruct the Proto-Tocharian form of each pair:
%
\tochmore {tree}{red}{\lettword}

\assignment
It is thought that Tocharian B had stress
(as in English more or less).
Upon what might this hypothesis be based?

\comment
\wipa{\A} is a prolonged \wipa a, \wipa{\d s} sounds as \word{sh},
\wipa{n5} as \word{ng};
the sequence \wipa{ts} is pronounced as a single consonant,
\wipa{\schwa} is a specific Tocharian vowel.
\by{(Svetlana Burlak)}

\newpage
\problem {30}
%
When describing how personal and reflexive pronouns work in various languages,
linguists make use of the so-called subscripts---Roman letters
(typically $i$, $j$, $k$, \dots)\ which mark pronouns and some other words
in sentences. The character $*$ (asterisk) is also used.
Here are some English examples:

\begin{enumerate}
\item John$_i$ saw himself$_i$ in the mirror.
\item John$_i$ says that he$_{i/j/{}^*k}$ doesn't know Peter$_k$.
\item The boy$_i$ is playing with his$_{i/j}$ gun.
\item His$_i$ teacher$_j$'s influence in easily seen in his$_{i/{}^*j/k}$ work.
\item The girl$_i$ saw her$_{{}^*i/j}$.
\end{enumerate}

\assignment Explain the meaning of the subscripts and the asterisk.

\assignment Add subscripts (and asterisks where appropriate)
in the following sentences:

\begin{items}
\item She doesn't like this trait in herself.
\item The father took his son to his room.
\item John knows that Peter has given his book to his son.
\end{items}
\by{(Maria Rubinstein)}

\problem {35}
%
Consider the following pairs of verbs with closely related meanings:

\medskip
\begin{tabular}{ll}
\word{accuse} & \word{rebuke} \\
\word{denounce} & \word{reprehend} \\
\word{command} & \word{instruct} \\
\word{advise} & \word{guide} \\
\word{assure} & \word{convince} \\
\end{tabular}
\medskip \\
%
It is known that all verbs in the left-hand column have a certain ability
that the verbs in the right-hand column lack.

\assignment Identify the ability in question.

\assignment Find the verbs that also have this ability among the following:
\word{extort}, \word{threaten}, \word{forbid}, \word{swear}, \word{shout},
\word{approve}, \word{refuse}, \word{rob}, \word{dedicate},
\word{lose}, \word{scold}, \word{give up}, \word{demand}.

\assignment Try to find two more verbs with the same ability.
\by{(Boris Iomdin)}

\editrans
\makepart{个人赛题解}

\solution

\begin{enumerate}
\item Nouns:
\begin{itemize}
\item \linzglos {work}
\begin{itemize}
\item Combinations: \linzglot {husband + wife}{brother + sister}
\item Family members are singled out by division and cancellation:
\linzglou {father}{brother}
\item Missing (deceased) family members are preceded by a minus sign:
\orphform\ = orphans' (apparently orphaned children of one and the same family).
\end{itemize}
\item $\It$ `person', $(>\It)$ `giant'.
\end{itemize}
\item Pronouns are composed of the character $\It$ or $\She$ (for feminine gender)
and the subscripts $1$ to $3$, which indicate the person.
\item The plural of nouns and pronouns is expressed by the coefficient $n$.
The plus sign plays the part of the conjunction `and'.
\item Verbs: $\talk$ `talk', $\work$ `work', $t$ `hurry',
$\scribe$ `write', $\heart$ `like, love', $\eat$ `eat'.
If what the verb denotes is absent or uncharacteristic, a minus sign expresses that:
$-\heart$ `not inclined to affection = wicked'.
(We can assume that a characteristic property is expressed by a plus sign,
hence $+\heart$ `good', a concept we need.)
\item Sentence structure:
\begin{itemize}
\item the subject is the base of the power;
\item the predicate is the exponent,
whereby negation is expressed by a minus sign ($-\heart$ `not like')
and passive voice by a radical sign ($\sqrt{\scribe}$ `be written');
additional activities can be added or subtracted
($\It_3^{\work-t}$ `he is working and doesn't hurry = he is working without haste');
\item past tense is marked by $-t$ ($\It_3^\work-t$ `he worked'),
future tense by $+t$;
\item the direct object, if there is one, follows an equals sign.
\end{itemize}
\end{enumerate}

\setcounter{rowcount}8
\assignment
%
\begin{tabular}[t]{rl}
\birow{He loves with an unrequited love (\emph{i.~e.}\ loves without being loved).}
\birow{The taciturn (\emph{or} mute) daughter will write about the father and the mother.}
\birow{You (sg.\ fem.)\ worked quickly (\emph{or} hastily) and silently.}
\birow{The letter was eaten by the hungry sister.}
\end{tabular}

\assignment \linzdone

\solution
%
本题所有的阿拉伯单词均符合以下模式之一:\wipa{1a2a3t}、\wipa{i12\A3}、\wipa{1u23} 火 \wipa{1u23\E n}
(符合一二模式的单词均按此顺序出现,而符合三四模式的单词均独立出现)。
在这些模式中,\wipa{1-2-3} 是以下辅音三元组之一:
\wipa{r-b-C}、\wipa{s-b-C}、\wipa{s-d-s}、\wipa{t-l-t}、
\wipa{t-m-n}、\wipa{t-s-C}、\wipa{x-m-s}、\wipa{C-\sh-r}。
让我们假设每个辅音三元组对应一个一与十之间的数字,以及元音的特定排列起到特定的功能:
\wipa{1a2a3t} \wipa{i1$'$2$'$\A3$'$} 是 $\frac n{n'}$ 或 $\frac {n'}n$
(不管哪种情况,$\egar{xamast ixm\A s}=\frac nn=1$),
且 $\egar{1u23} = \frac in$、$\egar{1u23\E n} = \frac jn$,虽然 $i$ 与 $j$ 仍然未知。

由等式 (5) 我们可以发现 \wipa{s-b-C} 和 \wipa{x-m-s} 是 $5$ 和 $7$(两种对应顺序均有可能),
并且,由 $\frac j5+\frac j7=\frac{(7+5)j}{35}=\frac{24}{35}$ 可知,$j=2$,也就是说,$\egar{1u23\E n} = \frac 2n$。
由于 \wipa{1u23} 比 \wipa{1u23\E n} 更短,我们可以假设前者对应着一更基础的功能,即 $\frac 1n$。

由 (1) 可知,\wipa{t-l-t} is $3$(以及在阿拉伯分数中,分子在分母之前)。
由 (4) 可知,\wipa{t-m-n} 比 \wipa{s-b-C} 大一。
由 (3) 可知,$3\egar{s-d-s} = 2\egar{t-s-C}$。
因此,\wipa{t-s-C} 可被三整除。由于 $3$ 是 \wipa{t-l-t},
\wipa{t-s-C} 与 \egar{s-d-s} 分别是 $6$ 和 $4$ 或者 $9$ 和 $6$,
以及 \wipa{t-m-n}、\wipa{s-b-C} 和 \wipa{x-m-s} 分别是 $8$、$7$ 和 $5$。

我们还没有看等式 (2)。\wipa{s-d-s} 显然不等于 $4$
($\frac73 + \frac14 = \frac{31}{12}$ 并不能约为一个分子分母均小于等于十的分数),
因此,$\egar{s-d-s}=6$,且
$\frac73 + \frac16 = \frac{15}6 = \frac52 = \frac{10}4 = \egar{C-\sh-r}/\egar{r-b-C}$。

\assignment
%
\fracdone

\assignment
%
$\egar{rubC} + \egar{Ca{\sh}art its\A C} = \frac14 + \frac{10}9 = \frac{49}{36}$
,$\egar{sabaCt isd\A s} = \frac76$。
因此,$\sqrt{\egar{rubC} + \egar{Ca{\sh}art its\A C}} = \egar{sabaCt isd\A s}$;
或者,如果括号不算符号的话:
$\egar{rubC} + \egar{Ca{\sh}art its\A C} = (\egar{sabaCt isd\A s})^2$。

\newpage\solution
%
本题出现了两种中文表述:(I) 日期、月份、以及星期几;(II) 星期几、第几个星期、以及月份。
巴斯克语表述中,(I) 的语序是 \<月份> \<日期>,\<星期几>,(II) 的语序是 \<月份> \<第几个星期> \<星期几>。
最后一个单词以 \word{-a} 结尾,而其前面一个单词没有尾 \word{-a}
(除了单词 \word{hogeita},其在数词词组中表示 `20')。
元素 \word{-garren} 组成了序数词。
单词 \word{astea} 并不是一个星期几的名字(其中六个出现在例子 1--10 中,第七个出现在任务三)。
由于任务二提到了单词 `周',我们可以猜出这就是 \word{astea} 的意思。

\assignment
%
\begin{tabular}[t]{ll}
\word{urtarrilaren hogeita hirugarrena, larunbata} & 一月二十三日,\CJKunderline{周六} \\
\word{abenduaren azken astea} & 十二月的最后\CJKunderline{一周} \\
\word{otsailaren lehenengo osteguna} & 二月的第一个周四 \\
\word{ekainaren bederatzigarrena, igandea} & 六月九日,周日 \\
\word{abenduaren lehena, \underline{asteazkena}} & 十二月一日,周三 \\
\word{irailaren azken asteazkena} & \CJKunderline{九月}的最后一个周三 \\
\word{azaroaren hirugarren ostirala} & 十一月的第三个周五 \\
\word{urriaren azken larunbata} & 十月的最后一个周六 \\
\word{irailaren lehena, astelehena} & 九月\CJKunderline{一日},周一 \\
\word{\underline{urtarrilaren} bigarrena, ostirala} & 一月二日,周五 \\
\end{tabular}

\assignment
%
\basqmore
{十二月的第一个周一}
{十一月二十九日,周六}
{一月的第二周}
{二月三日,周一}

\assignment
%
\word{Astelehena} `周一',\word{asteazkena} `周三';
\word{asteartea},唯一一个我们没有在任务一粒找到的星期几,是 `周二'。
三个名词都是由 \word{aste} `周' 构成的。
\word{Astelehena} 可直译为「一周中的第一(天)」,
\word{asteazkena} 可直译为「一周中的最后一(天)」,
星期二的巴斯克语大致可以译为「一周中间的日子」。

没人切确地知道为什么巴斯克人管星期三叫「一周中的最后一(天)」。
在巴斯克方言中存在星期几名词的其它几个变体,包括来自罗曼语族的外来词。

\newpage
\solution
%
The Adyghe sentences have the following structure:

\medskip
\adyganal {He $V$ & $X$ & $P$ $Y$}{What does he $V$ && $P$ $Y$}{Where does he $V$ & $X$}
\medskip \\
%
where $X$ and $Y$ are nouns, $V$ is a verb (or its stem)
and $P$ is, in English, one of the prepositions \word{into}, \word{onto} or \word{under}
and in Adyghe it is one of the prefixes \wipa{d-}, \wipa{tyr-} or \wipa{\d{\cj}-}.
As the third schema shows, the Adyghe locative prefix
may not correspond to anything in the natural (but imprecise) English translation.

\assignment
%
We specify (at the expense of naturalness):

\setcounter{rowcount}5
\medskip
\begin{tabular}{rl}
\birow{他把[这]碟子放在什么的下面?}
\birow{他把[这]碟子投在什么的上面?}
\end{tabular}

\assignment
%
\begin{tabular}[t]{rl}
\birow{他把[这]凳子投在[这]炉子里面。}
\birow{他把[这]钱落在什么的里面?}
\end{tabular}

\assignment \adygdone

\assignment
%
\adygmore {他把[这]桌子放在什么的里面?}{他把[这]桌子放在什么的上面?}{他把[这]桌子放在什么的下面?}

\vfill
\solution

\frendone \medskip \\
%
The table features verbs with two different prefixes: \word{re-} and \word{r\'e-}.
All verbs with \word{re-} indicate a repetition
or a renewal of the action named by the verb without a prefix.
Contrariwise, if the prefix is \word{r\'e-},
then the corresponding prefixless verb either doesn't exist
or means the same thing as the prefixed one does.
The verbs whose stems begin with vowels are an exception:
the prefix they take is \word{r\'e-}
regardless of the existence and the meaning of a corresponding prefixless verb.
There are other exceptions from this rule in French,
but on the whole it is fairly reliable.

\comment
The vowel in the prefix \word{r\'e-} is not unlike the first vowel in \word{raider},
whereas the one in the prefix \word{re-} bears a certain similarity to the second,
and needs to be fortified when it finds itself next to another vowel.

\editrans
\makepart{Solutions to the Problems of the Team Contest}

\solution
%
\assignment

\medskip
\tochdone
\medskip \\
%
The first pair gives the correspondence \wipa{\d st}---\wipa{st}.
This determines unambiguously the second pair (or triple, rather), whence we learn
that Tocharian B has kept the final vowels (except for the `specific' one) and Tocharian A has lost them.
Consequently all words  with retained final vowels are Tocharian B
and their counterparts with lost final vowels are Tocharian A.
This allows the following conclusions to be made:
In Tocharian A the `specific' vowel falls out before a vowel that is retained
and is retained before one that is lost;
\wipa a, long or short, is preserved without change.
In Tocharian B the `specific' vowel can become \wipa a, \wipa{\schwa} or nothing
and both {\wipa a\/}s can become either \wipa a or \wipa{\A}.
This determines the remaining pairs.

\assignment
%
(a) A \wipa{\d st{\A}m}, B \wipa{st{\A}m} `tree' $<$ *\wipa{st{\A}m{\schwa}};
(b) A \wipa{rt{\schwa}r}, B \wipa{ratre} `red' $<$ *\wipa{r{\schwa}t{\schwa}re};
(c) A \wipa{p{\schwa}rs}, B \wipa{parso} `\lettword' $<$ *\wipa{p{\schwa}rso}.
%
In the reconstruction the `specific' vowel is not inserted in clusters of the type
`sonant + obstruent' and the cluster \wipa{st},
nor is it added after final \wipa r.

\assignment
It is assumed that under stress *\wipa{\schwa} $>$ \wipa a, *\wipa{a/\A} $>$ long \wipa{\A},
whereas without stress *\wipa{\schwa} $>$ nothing or \wipa{\schwa} (as in Tocharian A),
*\wipa{a/\A} $>$ short \wipa a.

\solution
%
\assignment The subscripts mark the participants in the situation (the persons mentioned in the sentence).
Identical letters mean identical individuals, different letters mean different individuals.
In this way it is shown which pronoun can refer to which noun.
If a pronoun can refer to more than one noun,
all possible subscripts are given, separated by slashes.
If a pronoun can refer to an individual not mentioned in the sentence,
a letter is used that doesn't mark any other word in the same sentence
(e.~g., \word{he} in (2) may be someone other than John or Peter, let's say Bill, if he exists at all).
An asterisk next to a letter indicates
that the pronoun can't refer to the noun with this subscript.

\assignment
\begin{items}
\item She$_i$ doesn't like this trait in herself$_i$.
\item The father$_i$ took his$_{i/{}^*j/k}$ son$_j$ to his$_{i/j/k/l}$ room.
\item John$_i$ knows that Peter$_j$ has given his$_{i/j/l}$ book to his$_{i/j/k/l/m}$ son$_k$.
\end{items}

%\newpage
\solution
%
\assignment
%
The left column contains what are technically known as performative verbs.
(The concept of performativity was introduced in 1965 by the English philosopher John Austin.) 
They are different from other verbs in that
the action they name can be performed by their use, rather than simply described.
So the words \word{`I accuse you of murder'}
all by themselves constitute an accusation;
the words \word{`I denounce you as an impostor'}, a denunciation;
\word{`I command you to report to the headquarters at once'}, a command;
\word{`I advise you not to go there'}, advice;
\word{`I assure you that this problem is not so hard'}, assurance.
Performativity is a rather peculiar property;
as the statement of the problem shows,
even verbs with very similar meanings can differ in its presence or absence
(one can't very well say \word{`I hereby reprehend your cowardice'}
of \word{`I convince you that this is the correct solution'}).

\assignment
%
These are the verbs \word{forbid} (\word{`I forbid leaving the room before the class is over'}),
\word{swear} (\word{`I swear to cheat no more'}),
\word{approve} (\word{`I approve of your decision'}),
\word{refuse} (\word{`I refuse to try to solve this problem'}),
\word{dedicate} (\word{`I dedicate this book to my parents'}),
\word{give up} (\word{`I can't do this problem, I give up'}),
\word{demand} (\word{`I demand to be told how this problem is to be solved'}).

\assignment For example,
\word{thank} (\word{`I thank you for the clarification'}),
\word{congratulate} (\word{`I congratulate you on your success'}).

\editrans
\end{document}
